\documentclass[a4paper]{article}

%% Language and font encodings
\usepackage[english]{babel}
\usepackage[utf8x]{inputenc}
\usepackage[T1]{fontenc}


%% Sets page size and margins
\usepackage[a4paper,top=3cm,bottom=2cm,left=3cm,right=3cm,marginparwidth=1.75cm]{geometry}

%% Useful packages
\usepackage{amsmath}
\usepackage{graphicx}
\usepackage[colorinlistoftodos]{todonotes}
\usepackage[colorlinks=true, allcolors=blue]{hyperref}
\usepackage[hyphenbreaks]{breakurl}

\title{Video Streaming in Vehicular Networks}
\author{Patrick e Anderson}

\begin{document}
\maketitle 
\section{Resume}
We will study the video streaming for vehicular networks in the context of Smart Cities. Here we want to analyze and improve the main factors of the quality of service and experience for it. For such, we will study the configurations and the mechanisms of vehicular Networks with the FOG computing and match with the video streaming.

\section{Introduction}
We will approach here two current topics on the development of new telecommunication technologies. One of them is present on our daily routine for about ten years [1].\\
The video streaming, popularized with the growing of the social medias, big part of this due to Youtube [2], today we have other services like Netflix, that let users watch videos without keeping the data from videos on their devices, and still watch in a pratical way their movies, series and other content.\\
Nowadays we have another kind of media, the live video streaming [3]. This kind of service let a user share his content and other people watch it live, like the same thing that we have on televisions today, but with the whole communication among content producer and their consumers within the Internet.\\
The second point, more recently and tendentious, are the vehicular networks, maybe the biggest promise of Smart Cities [4], a theme that has acquired attention in the scientific field and we want to explore. It consist in the communication among cars, in a certain way, in real time, in a manner they could exchange information that would be sustained by a specific cloud. \\
With this form of exchange of information among cars and cloud servers the communication shall be quicker, cheaper and more efficient. This concept exist for us how QoS and involve  a series of concepts that will be look after.\\
Our focus here will be study a way to improve the QoE of this services, which refers a quality of experience of the users, and QoS of the transmission of this live streaming in a vehicular networks. Our motivation becomes from the growing usage of this kind of live platform [7, 8] and the notorious interest by general users to watch your videos, movies and series in everywhere, this way, our interest is to improve this experience for them.

\section{Network Challenges}
%%topicos a se pensar
%%veiculos em movimentos => moviementação dos servers,
%%troca servidor - servidor, servidor - carro (=carro-carr), servidor-cloud


%% Esse chap todo ta baseado no primeiro paper que lemos, que explica a rede e tal, que tem até o disktra. temos
%%que ver um jeito de referenciar o chap todo de uma vez

Now we can detail the specifics structure challenges of Vehicular Networks. Here we will describe this scenario pointing the principal terms and concepts, and after, treat two factors of the settings connection. \\
Basically, the Vehicular Networks are the union of trade information among cars intermediated for a specifics Networks servers and the communication among these servers. \\ 
To have quicker traffic of information within the Network it has Servers nearest to edge devices than the Cloud. \\
To contextualize , this objects for us are the \textit{nodes} of the Network and can be all devices that can connected to the Internet. This Servers, that stay in middle way between Cloud and cars, are the Cloudlets. The location of this Cloudlets we will define how the \textit{board} of Network, meaning that this Server are nearest of the nodes than Cloud. And finally, the union of the nodes, the Cloud and the Cloudlets we will define how a FOG. \\
So, the FOG computing is the way that we can study the connection of the nodes with the principal Server intermediated per Cloudlets that stay in the \textit{board} of the Network.

Due to the cars are moving we have to think in a way to keep this cars connected with the servers and don't lost information. In this hang, we have our two points of consideration of the structure of this scenario, the connection Cloudlet - Cloudlet and the connection Cloudlet - nodes (cars). \\
The first, connection Cloudlet - Cloudlet, should support the movement of the cars due to if a car are moving around a road, after of the connection establishment with the Cloudlet, this vehicle every will be distance of you connection point. Therefore the connection will be weakening and the trade of information will be more slow. So, a solution for this problem is the communication between the Cloudlets. \\
If exist many Cloudlets in the way around the road, or near sufficiently to be more near than the first connection point, the informations of it car can be trade between the Cloudlets around you route for allow that the server communication with it be quick due to the Cloudlet aways is near of the destination node.

The second, connection Cloudlet - Nodes (cars) consist in the communication among cars, in a certain way, in real time, in a manner they could exchange information that would be sustained, stored, transported and manipulated by equipment that are between the cloud and the vehicles.\\ 
With this, we can obtain informations how the proximity of others cars, validate the moves that this car can do, and, in our context, receive the streaming the video. \\
This kind of Network services will need to have a communication among them to be able to track a car in a certain path, minimizing the loss of packages, maintaining the QoS and still letting the cost of communication cheap [5, 6] loading it with a good quality with low bandwidth, such as we have on cars on roads nowadays [9].

\section{References}
[1] Infographic about the history of streaming video: https://www.ustream.tv/blog/streaming-video-tips/a-brief-history-of-streaming-video/
\\ \ [2] Bank of America analysis about importance of Youtube at the market: \\ https://www.bloomberg.com/news/articles/2015-05-27/a-bank-of-america-analysis-says-youtube-is-worth-more-than-85-percent-of-companies-in-the-s-p-500
\\ \ [3] Definition of a live stream and a broadcast: \\ 
https://developers.google.com/youtube/v3/live/broadcasts-and-streams
\\ \ [4] *Md Whaiduzzaman, Mehdi Sookhak, Abdullah Gani, and Rajkumar Buyya. A
survey on vehicular cloud computing. Journal of Network and Computer
Applications, 40:325–344, 2014.*
\\ \ [5] Shanhe Yi, Cheng Li, and Qun Li. A survey of fog computing: concepts,
applications and issues. In Proceedings of the 2015 Workshop on Mobile Big
Data, pages 37–42. ACM, 2015.
\\ \ [6]  *Tarik Taleb, Sunny Dutta, Adlen Ksentini, Muddesar Iqbal, and Hannu Flinck.
Mobile edge computing potential in making cities smarter. IEEE Communications
Magazine, 2016.*
\\ \ [7] Numbers about one of the most important live stream company, Twitch.tv: \\ https://www.twitch.tv/p/about
\\ \ [8] Amazon bought Twitch.tv: https://www.wsj.com/articles/amazon-to-buy-video-site-twitch-for-more-than-1-billion-1408988885
\\ \ [9] (*Seria bom uma referência comparando a banda de redes móveis com as caseiras.*)
\end{document}